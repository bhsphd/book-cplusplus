%!TEX program = xelatex
\documentclass[UTF8,aspectratio=43,10pt,t]{ctexbeamer}

\mode<presentation> {
\usetheme{Madrid}
\setbeamertemplate{footline}[frame number] %设置页码
\setbeamercolor{page number in head/foot}{fg=blue} %设置页码颜色
\setbeamertemplate{navigation symbols}{} %去除控件
}
\usepackage{indentfirst}
\setlength{\parindent}{2em}
\usepackage{listings}
\usepackage{wrapfig}
\usepackage{graphicx}
%设置图片存放路径
\graphicspath{{figures/}}

%用户自定义区块环境
\usepackage{setspace}
\definecolor{hanblue}{rgb}{0.27, 0.42, 0.81}
\definecolor{indiagreen}{rgb}{0.07, 0.53, 0.03}
\definecolor{indianred}{rgb}{0.8, 0.36, 0.36}
\definecolor{indianyellow}{rgb}{0.89, 0.66, 0.34}
\definecolor{babypink}{rgb}{0.96, 0.76, 0.76}
\definecolor{ao(english)}{rgb}{0.0, 0.5, 0.0}
\setbeamerfont{block title}{size=\small}
\setbeamerfont{block body}{size=\footnotesize}
\newenvironment<>{blueblock}[1]{
    \setbeamercolor{block title}{fg=white,bg=hanblue}
    \begin{block}#2{#1}}{\end{block}}
\newenvironment<>{greenblock}[1]{
    \setstretch{1.3}\setbeamercolor{block title}{fg=white,bg=indiagreen}
    \begin{block}#2{#1}}{\end{block}}
\newenvironment<>{redblock}[1]{
    \setstretch{1.3}\setbeamercolor{block title}{fg=white,bg=indianred}
    \begin{block}#2{#1}}{\end{block}}
\newenvironment<>{yellowblock}[1]{
    \setstretch{1.3}\setbeamercolor{block title}{fg=white,bg=indianyellow}
    \begin{block}#2{#1}}{\end{block}}

\lstset{language=C++,
columns=flexible,
basicstyle=\footnotesize\ttfamily,                                      % 设定代码字体、大小
%numbers=left,xleftmargin=2em,framexleftmargin=2em,                   % 在左侧显示行号
%numberstyle=\color{darkgray},                                        % 设定行号格式
keywordstyle=\color{blue},                                            % 设定关键字格式
commentstyle=\color{ao(english)},                                     % 设置代码注释的格式
stringstyle=\color{brown},                                            % 设置字符串格式
%showstringspaces=false,                                              % 控制是否显示空格
%frame=lines,                                                         % 控制外框
breaklines,                                                           % 控制是否折行
postbreak=\space,                                                     % 控制折行后显示的标识字符
breakindent=5pt,                                                      % 控制折行后缩进数量
emph={size\_t,array,deque,list,map,queue,set,stack,vector,string,pair,tuple}, % 非内置类型
emphstyle={\color{teal}},
escapeinside={(*@}{@*)},
}

\title[\textit{C++程序设计:第十二章}]{第十二章~工具与技术}

% \author
%     []
%     {}

\date {}

% \institute
%     {}


\begin{document}

\maketitle

\begin{frame}
    {目录}
    \tableofcontents
\end{frame}

\begin{frame} {前言}
    \begin{yellowblock}{学习目标}
        \begin{enumerate}
            \item 理解并掌握命名空间的使用方法;
            \item 掌握异常处理的使用方法;
            \item 理解和使用多重继承以及虚继承;
            \item 了解嵌套类和运行时类型识别的使用方法;
            \item 学会使用标准库中一些特殊工具,包括tuple类型、bitset类型以及对日期和时间的处理。
        \end{enumerate}
    \end{yellowblock}
\end{frame}

\section{命名空间}
\begin{frame}[fragile]{命名空间}
    将大量由于共同开发等原因使用的全局名字引入到同一作用域时,不可避免会产生命名冲突,例如:
    \begin{columns}[T]
        \column{0.65\textwidth}
        \begin{blueblock}{定义相同名称的不同函数}
            \begin{lstlisting}[moreemph={T}]
//Foo.h
int doSomething(int x, int y) {
    return x * y;
}
//Goo.h
int doSomething(int x, int y) {
    return x + y;
}
//main.cpp
#include "Foo.h"
#include "Goo.h"
int main() {
    int x = doSomething(2, 1); //错误:doSomething已经被定义
}
            \end{lstlisting}
        \end{blueblock}
        \column{0.3\textwidth}
        \begin{yellowblock}
            {说明}
            $\bullet$ 用户先在不同头文件里定义了同名的doSomething函数\\
            $\bullet$ main.cpp文件中的代码在引用了均包含doSomething函数的不同头文件后,试图调用doSomething,引发编译错误\\
        \end{yellowblock}
    \end{columns}
\end{frame}

\subsection{定义命名空间}
\begin{frame}[fragile]{命名空间}{定义命名空间}
    \begin{block}
        {命名空间}
        命名空间可以将全局作用域内有效的类名、函数名或对象名组织到一个名字下面。即将全局作用域分割为子作用域,每个子域称为一个命名空间。
    \end{block}
    \begin{columns}[T]
        \column{0.65\textwidth}
        \begin{blueblock}{命名空间示例}
            \begin{lstlisting}[moreemph={T}]
//Foo.h
namespace Foo {
int doSomething(int x, int y);
class X{ /*...*/ };
}
            \end{lstlisting}
        \end{blueblock}
        \column{0.3\textwidth}
        \begin{yellowblock}
            {说明}
            $\bullet$ 定义一个命名空间以关键字namespace开始,后面跟命名空间的名字\\
            $\bullet$ 主体由一对花括号括起来的声明和定义组成\\
            $\bullet$ 左侧代码定义了一个名为Foo的命名空间,该命名空间包括两个成员:一
            个函数和一个类\\
        \end{yellowblock}
    \end{columns}
\end{frame}

\begin{frame}[fragile]{命名空间}{定义命名空间}
    一个命名空间是可以不连续的。通常情况下,我们将命名空间成员声明放到头文件中,实现放在源文件,从而达到接口和实现分离的目的。例如:

    \begin{columns}[T]
        \column{0.65\textwidth}
        \begin{blueblock}{命名空间示例}
            \begin{lstlisting}[moreemph={T}]
//Foo.h
namespace Foo {
int doSomething(int x, int y);
class X{ /*...*/ };
}

//Foo.cpp
namespace Foo {
X g_x;
int doSomething(int x, int y) { /*...*/ }
}
            \end{lstlisting}
        \end{blueblock}
        \begin{blueblock}{外部访问命名空间}
            \begin{lstlisting}[moreemph={T}]
int x = Foo::doSomething(2, 1);
            \end{lstlisting}
        \end{blueblock}
        \column{0.3\textwidth}
        \begin{yellowblock}
            {说明}
            $\bullet$ 已经在头文件Foo.h中定义了命名空间Foo\\
            $\bullet$ Foo.cpp 源文件中,打开命名空间Foo,同时为它新增成员g\_x以及函数doSomething的实现\\
        \end{yellowblock}
    \end{columns}
\end{frame}

\begin{frame}[fragile]{命名空间}{定义命名空间}
    在命名空间外部访问它内部的成员时,必须要明确指出成员所属的命名空间:
    \begin{columns}[T]
        \column{0.65\textwidth}
        \begin{blueblock}{外部访问命名空间}
            \begin{lstlisting}[moreemph={T}]
int x = Foo::doSomething(2, 1);
            \end{lstlisting}
        \end{blueblock}
        \column{0.3\textwidth}
        \begin{yellowblock}
            {说明}
            $\bullet$ 调用命名空间Foo中的成员doSomething\\
        \end{yellowblock}
        \onslide<2->{
            \begin{redblock}
                {提示}
                一个命名空间可以定义在全局作用域内,也可以定义在其它命名空间内,但不能定义在
                函数或类的内部。
            \end{redblock}
        }
    \end{columns}
\end{frame}

\begin{frame}[fragile]{命名空间}{定义命名空间}
    \begin{block}
        {命名空间的嵌套}
        在一个命名空间内部定义另外一个命名空间
    \end{block}
    \begin{columns}[T]
        \column{0.65\textwidth}
        \begin{blueblock}{命名空间的嵌套示例}
            \begin{lstlisting}[moreemph={T}]
namespace Wang {
    namespace Goo {
        int doSomething(int x, int y);
    }
    namespace Boo {
        class Y {/*...*/};
    }
}
            \end{lstlisting}
        \end{blueblock}
        \begin{blueblock}{嵌套命名空间的访问}
            \begin{lstlisting}[moreemph={T}]
int x = Wang::Goo::doSomething(2, 1);
            \end{lstlisting}
        \end{blueblock}
        \column{0.3\textwidth}
        \begin{yellowblock}
            {说明}
            $\bullet$ 上面的代码在命名空间Wang 的内部分别定义了另外两个命名空间:Goo 和Boo\\
            $\bullet$ 要访问内层命名空间的名字,必须要使用嵌套的命名空间名字\\
        \end{yellowblock}
    \end{columns}
\end{frame}

\begin{frame}[fragile]{命名空间}{定义命名空间}
    C++11 标准新增\textbf{内联命名空间(inline namespace)},用来指示命名空间中的名称可以\alert{在外层命名空间中直接使用}。当一个程序的新版本发布时,我们使用内联命名空间,例如:
    \begin{columns}[T]
        \column{0.65\textwidth}
        \begin{blueblock}{内联命名空间示例}
            \begin{lstlisting}[moreemph={T}]
namespace FirstVersion {
    void fun(int);
}
inline namespace SecondVersion {
    void fun(int);
    void fun(double);
}

FirstVersion::fun(1); //调用早期版本fun函数
fun(1); //调用当前版本fun函数
fun(1.0); //调用当前版本中新增的fun 函数
            \end{lstlisting}
        \end{blueblock}
        \column{0.3\textwidth}
        \begin{yellowblock}
            {说明}
            $\bullet$ 关键字inline 必须出现在一个命名空间第一次定义的地方\\
            $\bullet$ 命名空间SecondVersion是内联的,因此它的成员可以在外层作用域(全局作用域)直接访问\\
            $\bullet$ 如果要访问早期版本(非内联)的成员,则必须要指明所属的版本名字\\
        \end{yellowblock}
    \end{columns}
\end{frame}

\begin{frame}[fragile]{命名空间}{定义命名空间}
    定义在全局作用域中的名字也是定义在\textbf{全局命名空间}中的。我们可以直接使用\alert{作用域操作符}访问全局命名空间的成员:
    \begin{columns}[T]
        \column{0.65\textwidth}
        \begin{blueblock}{访问全局命名空间成员}
            \begin{lstlisting}[moreemph={T}]
::member_name
            \end{lstlisting}
        \end{blueblock}
        \column{0.3\textwidth}
        \begin{yellowblock}
            {说明}
            $\bullet$ 全局命名空间是隐式声明的,每个文件将全局作用域内定义的名字添加到全局命名空间中。\\
        \end{yellowblock}
    \end{columns}
\end{frame}

\subsection{使用命名空间}
\begin{frame}[fragile]{命名空间}{使用命名空间}
    为了简化如下的繁琐的命名空间成员的访问方式,我们可以使用\textbf{using声明(using declaration)}或\textbf{using指示(using directive)}
    \begin{blueblock}
        {繁琐的命名空间成员访问形式}
        \begin{lstlisting}[moreemph={T}]
Wang::Goo::doSomething
        \end{lstlisting}
    \end{blueblock}
\end{frame}

\begin{frame}[fragile]{命名空间}{使用命名空间}
    一条\textbf{using声明}语句用来引用命名空间中的\alert{一个成员}。例如,我们使用using 声明引入标准库命名空间std中成员cout:
    \begin{columns}[T]
        \column{0.65\textwidth}
        \begin{blueblock}
            {使用using声明}
            \begin{lstlisting}[moreemph={T}]
using std::cout; //此声明告诉编译器后续cout属于命名空间std
cout << "Hello world"; //cout等价于std::cout
        \end{lstlisting}
        \end{blueblock}
        \column{0.3\textwidth}
        \begin{yellowblock}
            {说明}
            $\bullet$ using声明引入的名字的作用域从声明的地方开始,直到using声明所在的作用域结束处为止\\
            $\bullet$ 左侧的using声明表明在其作用域范围内,所有的cout都是指std::cout\\
        \end{yellowblock}
    \end{columns}
\end{frame}

\begin{frame}[fragile]{命名空间}{使用命名空间}
    using 声明一次只能引入命名空间的一个成员,如果要引入一个命名空间内\alert{所有的成员},我们可以使用\textbf{using指示}:
    \begin{columns}[T]
        \column{0.65\textwidth}
        \begin{blueblock}
            {使用using指示}
            \begin{lstlisting}[moreemph={T}]
using namespace std;
        \end{lstlisting}
        \end{blueblock}
        \column{0.3\textwidth}
        \begin{yellowblock}
            {说明}
            $\bullet$ using指示意味着std中所有成员在此处都可见\\

        \end{yellowblock}
        \only<1>{\begin{redblock}
                {注意}
                $\bullet$ 指示可以出现在全局作用域、局部作用域和命名空间作用域中,但是不能出现在类的作用域中\\
                $\bullet$ using指示的作用域也是从声明的地方开始,直到using语句所在的作用域结束处为止\\
            \end{redblock}}
    \end{columns}
    \only<2>{\begin{redblock}
            {提示}
            虽然using 声明只能引入命名空间中的一个成员,但与using指示相比,它\alert{不易引起命名冲突},是一种更安全的方式。
        \end{redblock}}
\end{frame}

\section{异常处理}
\begin{frame}[fragile]{异常处理}
    我们很难保证一个大型程序在运行期间不会出现错误,如果出现了错误,程序很可能无法正确运行,甚至会崩溃。\textbf{异常处理(exception handling)}允许我们将\alert{异常检测和解决的过程分离开来},程序中某一个模块出现了异常不会导致整个程序无法正确运行。
    C++ 语言提供了异常内部处理机制,该处理机制涉及到三个关键字:
    \begin{itemize}
        \item try:检测可能产生异常的语句块
        \item catch:捕获异常
        \item throw:抛出异常
    \end{itemize}
\end{frame}

\subsection{抛出异常}
\begin{frame}[fragile]{异常处理}{抛出异常}
    当程序在运行期间\alert{出现异常}时,我们可以通过throw来\alert{抛出}一个异常。例如,以下函数返回a除以b的结果,如果出现除数为0的情况:
    \begin{columns}[T]
        \column{0.65\textwidth}
        \begin{blueblock}
            {抛出一个异常}
            \begin{lstlisting}[moreemph={T}]
double divide(int a, int b){
    if (b == 0)
        throw "Error, division by zero!";
    return a / b;
}
        \end{lstlisting}
        \end{blueblock}
        \begin{blueblock}
            {抛出各种类型的异常}
            \begin{lstlisting}[moreemph={T}]
throw -1; // 抛出一个整型数
throw x; // x为double类型对象
throw MyException("Fatal Error"); //MyException为一个类类型
        \end{lstlisting}
        \end{blueblock}
        \column{0.3\textwidth}
        \begin{yellowblock}
            {说明}
            $\bullet$ throw可以抛出\alert{任何类型对象}\\
            $\bullet$ 通常情况下,抛出的异常为错误的编号、错误描述或用户自定义的异常类对象。\\
            $\bullet$ 当执行throw语句时,其\alert{后面的语句不会被执行}。程序的控制权将转移到\alert{与之匹配的catch模块}\\
        \end{yellowblock}
    \end{columns}
\end{frame}

\subsection{检测异常}
\begin{frame}[fragile]{异常处理\small{——检测异常}}
    C++ 语言通过\textbf{关键字\texttt{try}}来检测可能发生异常的代码。通常情况下,我们\alert{将可能发生异常的代码放到\texttt{try}语句块中},该语句块中的任何异常都可以被检测到。
    \begin{columns}[T]
        \column{0.65\textwidth}
        \begin{blueblock}
            {检测异常}
            \begin{lstlisting}[moreemph={T}]
try{
    divide(a, b); //函数调用语句
}
        \end{lstlisting}
        \end{blueblock}
        \column{0.3\textwidth}
        \begin{yellowblock}
            {说明}
            $\bullet$ 一旦在try 语句块内部有异常抛出时,系统检查与该try块关联的catch子句,并寻找\alert{与异常相匹配的catch子句}。\\
        \end{yellowblock}
    \end{columns}
\end{frame}

\begin{frame}[fragile]{异常处理\small{——捕获异常}}
    最终,我们通过\textbf{catch子句}\alert{捕获异常},并处理它:
    \begin{columns}[T]
        \column{0.65\textwidth}
        \begin{blueblock}
            {检测异常}
            \begin{lstlisting}[moreemph={T}]
catch (const char *str) {//捕获一个C风格字符串常量对象
    //任何能够被char *接受的异常都将被捕获
    cerr << "捕获异常" << str << endl; //cerr为标准错误ostream对象
}
        \end{lstlisting}
        \end{blueblock}
        \column{0.3\textwidth}
        \begin{yellowblock}
            {说明}
            $\bullet$ cerr为标准错误ostream对象用于输出程序错误信息。\\
            $\bullet$ catch语句中的异常声明类似于只包含一个形参的函数形参列表。\\
            $\bullet$ 异常声明包括类型和名字,其中类型决定了该catch子句能够捕获的异常的类型\\
            $\bullet$ 能够捕获的错误类型可以为左值引用,但不能为右值引用。\\
        \end{yellowblock}
    \end{columns}
\end{frame}


\subsection{捕获异常}
%!可能要有图
\begin{frame}[fragile]{异常处理\small{——捕获异常}}
    一个包含try、catch和throw的异常处理案例如下:\\
    \begin{columns}[T]
        \column{0.65\textwidth}
        \begin{blueblock}
            {异常处理案例}
            \begin{lstlisting}[moreemph={T}]
int a = 1, b = 0;
try {
    int c = divide(a, b);
}
catch (const string &str) {
    cerr << str << endl;
}
catch (const char *str) {
    cerr << str << endl;
}
        \end{lstlisting}
        \end{blueblock}
        \column{0.3\textwidth}
        \begin{yellowblock}
            {说明}
            异常被抛出后:
            \begin{enumerate}
                \item try后面的catch尝试匹配,匹配则处理
                \item 否则沿着调用链向外层逐层检查
                \item 无法匹配,则调用terminate终止程序
            \end{enumerate}
        \end{yellowblock}
        \begin{greenblock}
            {问题}
            异常被谁捕获?
        \end{greenblock}
    \end{columns}
    \begin{redblock}
        {注意}
        通常情况下,异常的类型和catch声明的类型要求严格匹配,但不包括以下情况:1)非常量到常量的转换;2)派生类向基类的转换;3)数组或函数被转换成指向数组元素或函数的指针。
    \end{redblock}
\end{frame}

\subsection{使用标准库异常类}
\begin{frame}[fragile]{异常处理\small{——标准库异常类}}
    C++标准库提供了\textbf{标准异常类(standard exception)},使用时需要包含头文件exception。
    其继承关系如图:
    \begin{figure}
        \includegraphics[scale=0.4]{Fig12-1.png}
    \end{figure}
    \begin{block}
        {基类exception}
        基类exception只定义了默认的构造函数、复制构造函数、赋值运算符、虚析构函数和一个名为\textbf{what}的虚成员。
    \end{block}
\end{frame}

\begin{frame}[fragile]{异常处理\small{——标准库异常类}}
    \textbf{what函数}返回一个const char*,指向一个以null结尾的字符数组,用于提示异常类型。我们可以继承exception类并重载what:\\
    \begin{columns}[T]
        \column{0.65\textwidth}
        \begin{blueblock}
            {自定义版本的what成员}
            \begin{lstlisting}[moreemph={T}]
struct MyException :public exception {
    const char* what() const noexcept { return "Ooops!"; }
};
        \end{lstlisting}
        \end{blueblock}
        \column{0.3\textwidth}
        \begin{block}
            {noexcept}
            C++11标准引入的新关键字,用来指明某个函数\alert{不会抛出异常}
        \end{block}
        \only<1>{\begin{yellowblock}
                {说明}
                noexcept应置于:\\
                $\bullet$ 形参列表后面\\
                $\bullet$ 如修饰成员函数,在const限定符之后,final、override或纯虚函数=0之前\\
                $\bullet$ 函数的声明和定义处必须都要出现\\
            \end{yellowblock}}
        \only<2->{
            \begin{redblock}
                {提示}
                noexcept说明可以优化代码的执行效率。
            \end{redblock}
        }
    \end{columns}
\end{frame}

\begin{frame}[fragile]{异常处理\small{——标准库异常类}}
    下面的代码将抛出一个MyException异常对象,该对象可以被异常声明为\alert{基类exception}类型的catch子句捕获:\\
    \begin{columns}[T]
        \column{0.65\textwidth}
        \begin{blueblock}
            {使用MyException对象}
            \begin{lstlisting}[moreemph={T}]
try {
    throw MyException();
}
catch (exception &ex) {
    cerr << ex.what() << endl;
    }
\end{lstlisting}
        \end{blueblock}
        \column{0.3\textwidth}
    \end{columns}
\end{frame}

\section{多重继承与虚继承}
\subsection{多重继承}
\begin{frame}[fragile]{多重继承与虚继承\small{——多重继承}}
    \begin{block}
        {多重继承}
        为一个派生类指定多个基类的继承结构称为多重继承
    \end{block}
    \begin{columns}[T]
        \column{0.65\textwidth}
        \begin{blueblock}
            {多重继承}
            \begin{lstlisting}[moreemph={T}]
class Mammal {
public:
    virtual void feedMilk() {} //母乳喂养
};
class WingedAnimal {
public:
    virtual void flap() {} //振翅飞翔
};
class Bat: public Mammal, public WingedAnimal { };
        \end{lstlisting}
        \end{blueblock}
        \begin{blueblock}
            {多重继承}
            \begin{lstlisting}[moreemph={T}]
Bat b;
b.feedMilk(); //母乳喂养
b.flap(); //振翅飞翔
        \end{lstlisting}
        \end{blueblock}
        \column{0.3\textwidth}
        \begin{yellowblock}
            {说明}
            $\bullet$ Bat类的派生列表中有两个以逗号分隔的基类\\
            $\bullet$ Bat类对象将具有Mammal和WingedAnimal两种动物的行为\\
        \end{yellowblock}
    \end{columns}
\end{frame}

\begin{frame}[fragile]{多重继承与虚继承\small{——多重继承}}
    \begin{columns}[T]
        \column{0.65\textwidth}
        \begin{blueblock}
            {多重继承}
            \begin{lstlisting}[moreemph={T}]
class Bat: public Mammal, public WingedAnimal { };
        \end{lstlisting}
        \end{blueblock}
        \begin{blueblock}
            {调用基类构造函数}
            \begin{lstlisting}[moreemph={T}]
Bat::Bat() {} //隐式调用Mammal和WingedAnimal的默认构造函数
Bat::Bat() :Mammal(), WingedAnimal() {} //显式调用基类的默认构造函数
        \end{lstlisting}
        \end{blueblock}
        \column{0.3\textwidth}
        \begin{yellowblock}
            {说明}
            $\bullet$ 多重继承的派生类对象的构造函数只能初始化其\alert{直接基类}成员\\
            $\bullet$ 构造的顺序与\alert{派生列表}中基类出现的先后顺序一致\\
            $\bullet$ 调用基类的构造函数有\alert{隐式和显式}两种方式\\
            $\bullet$ 成员析构的顺序与构造的顺序相反\\
        \end{yellowblock}
    \end{columns}
\end{frame}

\subsection{虚继承}
\begin{frame}[fragile]{多重继承与虚继承\small{——虚继承}}
    我们将Mammal和WingedAnimal进一步抽象,设计一个公共基类Animal:
    \begin{columns}[T]
        \column{0.65\textwidth}
        \begin{blueblock}
            {加入公共基类Animal}
            \begin{lstlisting}[moreemph={T}]
class Animal {
protected:
    int m_age;
public:
    Animal(int n = 0) :m_age(n) {}
    virtual void eat() {}
};
class WingedAnimal: public Animal{
public:
    virtual void feedMilk() {}
};
class Mammal: public Animal{
public:
    virtual void flap() {}
};
class Bat: public Mammal, public WingedAnimal { };
        \end{lstlisting}
        \end{blueblock}
        \column{0.3\textwidth}
        \begin{figure}
            \includegraphics[scale=0.6]{Fig12-2.png}
        \end{figure}
        \begin{redblock}
            {死亡钻石}
            菱形继承关系造成的二义性问题
        \end{redblock}
    \end{columns}
\end{frame}

\begin{frame}[fragile]{多重继承与虚继承\small{——虚继承}}
    当存在如下调用的时候,会产生二义性问题:
    \begin{columns}[T]
        \column{0.65\textwidth}
        \begin{blueblock}
            {加入公共基类Animal}
            \begin{lstlisting}[moreemph={T}]
Bat b;
b.eat(); //错误:二义性访问
Animal a = b; //错误:类型无法转换
        \end{lstlisting}
        \end{blueblock}
        \column{0.35\textwidth}
    \end{columns}
\end{frame}

\begin{frame}[fragile]{多重继承与虚继承\small{——虚继承}}
    C++通过\textbf{虚继承(virtual inheritance)}的机制来解决上述问题:
    \begin{columns}[T]
        \column{0.65\textwidth}
        \begin{blueblock}
            {加入公共基类Animal}
            \begin{lstlisting}[moreemph={T}]
class WingedAnimal: virtual public Animal {/*...*/};
class Mammal : virtual public Animal {/*...*/};
        \end{lstlisting}
        \end{blueblock}
        \column{0.3\textwidth}
        \begin{yellowblock}
            {说明}
            $\bullet$ 通过在派生列表中添加关键字\alert{virtual}来指定虚基类\\
            $\bullet$ 不论该虚基类在继承体系中出现多少次,在派生类中只包含\alert{唯一一份}共享的虚基类成员\\
        \end{yellowblock}
    \end{columns}
\end{frame}

\begin{frame}[fragile]{多重继承与虚继承\small{——虚继承}}
    虚继承的基类由\alert{最底层的派生类进行初始化}:
    \begin{columns}[T]
        \column{0.65\textwidth}
        \begin{blueblock}
            {虚继承对象的构造}
            \begin{lstlisting}[moreemph={T}]
class Bat : public Mammal, public WingedAnimal {
public:
    Bat() :Animal(1), Mammal(), WingedAnimal(){}
};
        \end{lstlisting}
        \end{blueblock}
        \column{0.3\textwidth}
        \begin{yellowblock}
            {说明}
            此处初始化顺序如下:\\
            $\bullet$ Bat类的构造函数提供的初始化列表初始化Animal成员\\
            $\bullet$ 构造Mammal成员\\
            $\bullet$ 构造WingedAnimal成员\\
        \end{yellowblock}
    \end{columns}
\end{frame}

\section{12.4~嵌套类}
\begin{frame}[fragile]{嵌套类}
    \begin{block}
        {嵌套类}
        在一个类的内部定义的类
    \end{block}
    \begin{yellowblock}
        {说明}
        $\bullet$ 嵌套类是\alert{独立}的类,与外层类在语法上没有关联\\
        $\bullet$ 嵌套类和外层类的访问控制遵循\alert{普通类}之间的访问控制原则\\
    \end{yellowblock}
\end{frame}

\subsection{二维数组类}
\begin{frame}[fragile]{嵌套类\small{——二维数组类}}
    \begin{greenblock}
        {任务}
        实现一个二维数组类,该类可以像普通二维数组那样支持两个下标操作:
        \begin{lstlisting}[moreemph={T}]
int arr[2][2];
arr[0][0] = 1;
\end{lstlisting}
    \end{greenblock}

    \begin{greenblock}
        {存在的难点}
        C++仅支持一维下标操作符的重载:
        \begin{lstlisting}[moreemph={T}]
class Array2D{
    /*...*/
    int operator [][](/*...*/); //错误:C++没有运算符[][]
};
\end{lstlisting}
    \end{greenblock}

    \onslide<2>{\begin{yellowblock}
        {分析}
        \texttt{arr[0][0]}等价于\texttt{(arr[0])[0]}
    \end{yellowblock}}
\end{frame}



\subsection{二维数组类}
\begin{frame}[fragile]{嵌套类\small{——二维数组类}}
    \begin{blueblock}
        {代码清单12.1~二维数组类}
        \begin{lstlisting}[moreemph={T}]
template<typename T>
class Array2D {
private:
    class Array1D {
        ... //下页内容
    };
    size_t m_size; //第一维长度
    Array1D *m_arr; //元素类型为Array1D
    public:
    Array2D(size_t s1, size_t s2) :m_size(s1), m_arr(new Array1D[s1]) {
        for (int i = 0; i<m_size; i++) {
            m_arr[i].m_size = s2;
            m_arr[i].m_arr = new T[s2];
        }
    }
    ~Array2D() { delete[] m_arr; }
    Array1D & operator[](int idx) { return m_arr[idx]; }
    size_t size() { return m_size; }
};
        \end{lstlisting}
    \end{blueblock}
\end{frame}

\begin{frame}[fragile]{嵌套类\small{——二维数组类}}
    \begin{columns}[T]
        \column{0.65\textwidth}
    \begin{blueblock}
        {代码清单12.1~二维数组类}
        \begin{lstlisting}[moreemph={T}]
...//上页内容
class Array1D {
    friend class Array2D; //声明为友元类
public:
    ~Array1D() { delete[] m_arr; }
    T & operator[](int idx) { return m_arr[idx]; }
private:
    size_t m_size = 0; //第二维长度
    T *m_arr = nullptr;
};
...//上页内容
\end{lstlisting}
    \end{blueblock}
    \begin{blueblock}
        {使用二维数组}
        \begin{lstlisting}[moreemph={T}]
Array2D<int> arr(2, 2);
arr[0][0] = 1;
\end{lstlisting}
    \end{blueblock}
    \column{0.3\textwidth}
    \begin{yellowblock}
        {说明}
        $\bullet$ 使用嵌套类将Array1D的底层实现隐藏
    \end{yellowblock}
\end{columns}
\end{frame}

\subsection{通用计算器}
\begin{frame}[fragile]{嵌套类\small{——二维数组类}}
    9.3.5节中我们实现的计算器程序有以下的代码:
    \begin{columns}[T]
        \column{0.65\textwidth}
    \begin{blueblock}
        {计算器}
        \begin{lstlisting}[moreemph={T}]
/*...*/
unique_ptr<Operator> oo;
if (o == ’+’)
    oo = make_unique<Plus>();
else if (o == ’-’)
    oo = make_unique<Minus>();
/*...*/
\end{lstlisting}
    \end{blueblock}
    \column{0.3\textwidth}
    \begin{greenblock}
        {思考}
        如果我们想要为计算器添加新的运算符,我们要怎么做?
    \end{greenblock}
\end{columns}
\end{frame}


\begin{frame}[fragile]{嵌套类\small{——二维数组类}}
    为了有利程序扩展,我们希望\alert{根据运算符的名字来自动创建相应运算符类对象}。
    首先我们需要实现一个\textbf{类注册机制},我们将用如下数据结构来实现:
    \begin{columns}[T]
        \column{0.65\textwidth}
    \begin{blueblock}
        {类注册机制核心数据结构}
        \begin{lstlisting}[moreemph={T}]
map<char, function<unique_ptr<Operator>()>> ms_operator;
\end{lstlisting}
    \end{blueblock}
    \column{0.3\textwidth}
    \begin{block}
        {类注册机制}
        保存\alert{类名(字符串)}和\alert{类实例}获取方法的映射关系,使程序能够根据名称得到类的实例。
    \end{block}
    \begin{yellowblock}
        {说明}
        $\bullet$ map对象的关键字为char类型,用于根据字符调用对应的function对象
        $\bullet$ function对象将返回一个指向Operator对象的unique\_ptr,用于生成对应运算符对象
    \end{yellowblock}
\end{columns}
\end{frame}

\begin{frame}[fragile]{嵌套类\small{——二维数组类}}
    下面代码将自动注册Plus类:
    \begin{columns}[T]
        \column{0.65\textwidth}
    \begin{blueblock}
        {类注册机制核心数据结构}
        \begin{lstlisting}[moreemph={T}]
ms_operator.emplace(’+’,[]() { return make_unique<Plus>(); });
\end{lstlisting}
    \end{blueblock}
    \column{0.3\textwidth}
    \begin{yellowblock}
        {说明}
        $\bullet$ emplace函数用于向map插入一组char到Operator对象生成器的映射\\
        $\bullet$ lambda表达式用于初始化function对象
    \end{yellowblock}
\end{columns}
\vspace{0.5cm}
\end{frame}

\begin{frame}[fragile]{嵌套类\small{——二维数组类}}
之后即可根据用算符名字自动创建该运算符类对象:
\begin{columns}[T]
    \column{0.65\textwidth}
\begin{blueblock}
    {类注册机制核心数据结构}
    \begin{lstlisting}[moreemph={T}]
ms_operator.emplace(’+’,[]() { return make_unique<Plus>(); });
\end{lstlisting}
\end{blueblock}
\column{0.3\textwidth}
\end{columns}
\begin{columns}[T]
    \column{0.65\textwidth}
    
\begin{blueblock}
    {根据名字自动创建运算符}
    \begin{lstlisting}[moreemph={T}]
unique_ptr<Operator> oo = ms_operator[’+’]();
\end{lstlisting}
\end{blueblock}
\column{0.3\textwidth}
\begin{yellowblock}
    {说明}
    $\bullet$ 下标运算符调用function对象\\
    $\bullet$ function对象通过lambda表达式调用make\_unique\\
    $\bullet$ make\_unique将返回对应下标运算符接收的字符所对应的Operator的unique\_ptr\\
\end{yellowblock}
\end{columns}
\end{frame}

\begin{frame}[fragile]{嵌套类\small{——二维数组类}}
    为了方便用户注册,我们实现一个对象工厂(object factory):
    \begin{columns}[T]
        \column{0.65\textwidth}
    \begin{blueblock}
        {代码清单12.2 Operator类对象工厂1}
        \begin{lstlisting}[moreemph={T}]
class Factory{
public:
template<typename T>
struct RegisterClass {
RegisterClass(char opr) {
Factory::ms_operator.emplace(opr,[]{return make_unique<T>();});
}//在Factory静态成员ms_operator作用域范围内,可以直接访问
};
static unique_ptr<Operator> create(char opr) {
auto it = ms_operator.find(opr);
if (it != ms_operator.end())
return it->second(); //调用与opr相关联的lambda表达式
}
private://静态成员
static map<char, function<unique_ptr<Operator>()>> ms_operator;
};
\end{lstlisting}
    \end{blueblock}
    \column{0.3\textwidth}
    \begin{yellowblock}
        {说明}
        $\bullet$ emplace函数用于向map插入一组char到Operator对象生成器的映射
        $\bullet$ lambda表达式用于初始化function对象
    \end{yellowblock}
\end{columns}
\vspace{0.5cm}
\end{frame}

\begin{frame}[fragile]{嵌套类\small{——二维数组类}}
    为了方便用户注册,我们实现一个对象工厂(object factory):
    \begin{columns}[T]
        \column{0.65\textwidth}
    \begin{blueblock}
        {代码清单12.2 Operator类对象工厂2}
        \begin{lstlisting}[moreemph={T}]
#define REGISTRAR(T, Key) Factory::RegisterClass<T> reg_##T(Key);
map<char, function<unique_ptr<Operator>()>> Factory::ms_operator;
\end{lstlisting}
    \end{blueblock}
    \column{0.3\textwidth}
    \begin{yellowblock}
        {说明}
        $\bullet$ emplace函数用于向map插入一组char到Operator对象生成器的映射
        $\bullet$ lambda表达式用于初始化function对象
    \end{yellowblock}
\end{columns}
\vspace{0.5cm}
\end{frame}

\section{运行时类型识别}
\subsection{dynamic_cast运算符}
\subsection{typeid运算符}

\section{union类型}
\subsection{定义union类型}
\subsection{使用union类型}

\section{union类型标准库特殊工具}
\subsection{tuple类型}
\subsection{bitset类型}
\subsection{日期和时间}

\end{document}